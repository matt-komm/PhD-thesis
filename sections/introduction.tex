\chapter{Introduction}

\section{Unit convention}

The ``natural units'' (of particle physics\footnote{Other fields of physics can have their own ``natural units'' which may not coincide with the definition used here.}) are used throughout this thesis unless it is explicitly specified otherwise. These differ from the \newacronym{si}{SI}{International System of Units }\gls{si} by defining the natural constants as following:

\begin{itemize}
\item speed of light: $\mathrm{c}\equiv 1$;
\item Planck constant: $\hbar\equiv 1$;
\item electric permittivity: $\epsilon_{0}\equiv 1$;
\item Boltzmann constant: $k_\mathrm{B}\equiv 1$.
\end{itemize}

This changes amongst others the units of standard quantities as follows:

\begin{itemize}
\item spatial distance: $\big[\mathrm{m}\big]\rightarrow \big[1/\GeV\big]$;
\item time: $\big[\mathrm{s}\big]\rightarrow \big[1/\GeV\big]$;
\item mass: $\big[\mathrm{kg}\big]\rightarrow \big[\GeV\big]$;
\item energy: $\big[\mathrm{J}\big]\rightarrow \big[\GeV\big]$;
\item temperature: $\big[\mathrm{K}\big]\rightarrow \big[\GeV\big]$;
\item cross section: $\big[\mathrm{m}^{2}\big]\rightarrow \big[1/\GeV^{2}\big]$.
\end{itemize}

\todo{summation convention, metric}

\section{Publications}

