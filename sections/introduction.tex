\chapter*{Introduction}
\addcontentsline{toc}{chapter}{Introduction}


\hspace{0.025\textwidth}\parbox[t][][t]{0.94\textwidth}{
\small\setstretch{1.3}{\slshape
``But I don't want to go among mad people,'' Alice remarked. ``Oh, you can't help that,'' said the Cat: ``we're all mad here. I'm mad. You're mad.'' ``How do you know I'm mad?'' said Alice. ``You must be,'' said the Cat, ``or you wouldn't have come here.''}\hfill --- Lewis Carroll, Alice in Wonderland\\[0.6\baselineskip]
}

The \acrlong{sm} is a very successful theory in describing electroweak and strong interactions between point-like, fundamental particles. Its predictions are constantly challenged with experimental data but so far no significant deviations have been confirmed. 

The heaviest known elementary particle is the top quark whose mass is similar to a gold atom. It was discovered in 1995 by the \gls{cdf} and \gls{d0} experiments at the Tevatron (Fermilab, USA) through its production process via strong interactions leading to top quark/antiquark pairs. An experimentally more elusive production mode of top quarks predicted by the standard model occurs via electroweak interactions resulting in events containing only single top quarks. The detection of such events is more challenging due to the larger occurrence of background events mimicking its signature. It took 14 more years until the production of single top quarks was finally observed in 2009 by the \gls{cdf} and \gls{d0} experiments as well. With the start of the physics program at the \gls{lhc} (\gls{cern}, Switzerland) in 2009 the production of single top quarks can be studied for the first time in great detail using an unprecedented quantity of proton-proton collision events.

Single top quark events offer a unique opportunity to measure the properties of the electroweak theory and to test its predictions in the presence of such a heavy particle. For instance, measurements of the inclusive cross section of single-top-quark production allow to infer the modulus of the \gls{ckm} matrix element $\vtb$ whose value is not predicted by the standard model. Measurements of differential cross section on the other hand yield in-deep tests of the electroweak production mechanisms and of its coupling structure which in case of deviations may also provide hints towards physics beyond the standard model.

In this thesis, the production of single top quarks via $t$~channel is investigated.  Measurements of differential cross sections based on proton-proton collision data at \acrlong{cm} energies of 8 and 13~\TeV with the \gls{cms} experiment are presented. 

In 8~\TeV data the top quark polarization angle is studied which is defined between the lepton from the top quark decay and the spectator quark in the top quark rest frame. The top quark spin asymmetry, a quantity related to the top quark polarization, is extracted from the differential cross section. The asymmetry allows to test the coupling structure of the involved electroweak interaction. In the standard model a \glshere{va} coupling structure is predicted which permits only left-handed top quarks or right-handed top antiquarks to interact with a W~boson and a bottom quark. Hence, one expects that single top quarks in $t$~channel are produced with a high degree of polarization. On the other hand, a potential depolarization may occur through new physics beyond the standard model. In effective field theory their influence can be recast into anomalous couplings for which limits are derived using the measured spin asymmetry. Events containing single muons or electrons together with two or three jets are analyzed in this measurement. Two boosted decision trees are trained to define a signal-enriched region from which the differential cross section is inferred from data through a regularized unfolding procedure.

In the second part, an early measurement of differential cross sections as a function of the top quark transverse momentum and rapidity is presented using the first data recorded at a \acrlong{cm} energy of 13~\TeV with the \gls{cms} experiment. For this, the analysis strategy of the top quark polarization measurement has been significantly extended. The new strategy involves the estimation of the amount of signal events as a function of the unfolding observables from data through multiple maximum-likelihood fits. The resulting differential cross sections are compared to the predictions by various event generator programs.

The thesis is organized as follows. First a general introduction to the standard model is given in Ch.~\ref{ch:theory}. The phenomenology of the top quark, its properties and production mechanisms, relevant for this thesis, are introduced in Ch.~\ref{ch:top} with a particular emphasis on the electroweak coupling structure. The experimental setup consisting of the \gls{lhc} accelerator and the \gls{cms} experiment is outlined in Ch.~\ref{ch:exp} which is followed by a detailed description of the reconstruction of physics objects for analyses in Ch.~\ref{ch:reconstruction}. In Ch.~\ref{ch:technique} the employed analysis techniques are elaborated. The performed differential cross section measurements at 8 and 13~\TeV are detailed in Ch.~\ref{ch:polarization} and Ch.~\ref{ch:diff13} respectively. Before the thesis is concluded improvements for future differential single-top-quark measurements are investigated in Ch.~\ref{ch:prospects}.


\section*{Unit convention}
\addcontentsline{toc}{section}{Unit convention}

The ``natural units'' of particle physics are used throughout this thesis unless it is explicitly stated otherwise. These differ from the \glsmark{si} units and are derived by defining the natural constants as:

\begin{itemize}
\item speed of light: $\mathrm{c}\equiv 1$\,;
\item Planck constant: $\hbar\equiv 1$\,;
\item electric permittivity: $\epsilon_{0}\equiv 1$\,;
\item Boltzmann constant: $k_\mathrm{B}\equiv 1$\,.
\end{itemize}

This changes the units of the following quantities amongst others:

\begin{itemize}
\item length: $\big[\mathrm{m}\big]~\mapsto~ \big[\eV^{-1}\big]$\,;
\item time: $\big[\mathrm{s}\big]~\mapsto~ \big[\eV^{-1}\big]$\,;
\item mass: $\big[\mathrm{kg}\big]~\mapsto~ \big[\eV\big]$\,;
\item energy: $\big[\mathrm{J}\big]~\mapsto~ \big[\eV\big]$\,;
\item temperature: $\big[\mathrm{K}\big]~\mapsto~ \big[\eV\big]$\,.
\end{itemize}



