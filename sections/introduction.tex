\chapter*{Introduction}
\addcontentsline{toc}{chapter}{Introduction}

The \acrlong{sm} is a very successful theory in describing electroweak and strong interactions between fundamental particles. Its predictions are constantly challenged with experimental data but so far no significant deviations have been observed. 

The heaviest known elementary particle is the top quark whose mass is similar to an entire gold atom. It has been discovered in 1995 by the \gls{cdf} and \gls{d0} experiments at the Tevatron (Fermilab, USA) through its production process via strong interactions leading to top quark/antiquark pairs. An experimentally more elusive production mode of top quarks predicted by the standard model occurs via electroweak interactions resulting into events containing only single top quarks. The detection of such events is more challenging due to the larger occurrence of background events mimicking its signature. It took 14 more years until the production of single top quarks was finally observed as well in 2009 by the \gls{cdf} and \gls{d0} experiments. With the start of the physics program at the \gls{lhc} (\gls{cern}, Switzerland) in 2009, the production of single top quarks can be studied for the first time in great detail utilizing an unprecedented quantity of proton-proton collision events.

Single top quark events offer an unique opportunity to measure the properties of the underlying electroweak theory and to test its predictions in the presence of such a heavy particle. For instance, measurements of the inclusive cross section of single-top-quark production allow to infer the \gls{ckm} matrix element $|\vtb|$ whose value is not predicted by the standard model. Measurements of differential cross section on the other hand allow to perform in-deep tests of the electroweak production mechanisms.

In this thesis, the production of single top quarks via $t$~channel is investigated.  First measurements of differential cross sections at \acrlong{cm} energies of 8 and 13~\TeV, based on data recored with the \gls{cms} experiment, are presented. 

At 8~\TeV, the top quark polarization angle is studied which is defined between the lepton from the top quark decay and the spectator quark in the top quark rest frame. It allows to probe the predicted V-A coupling structure in electroweak interactions which permits only left-handed top quarks or right-handed top antiquarks to interact with a W~boson and a bottom quark. This results into the production of single top quarks which are highly polarized along the spectator quark axis in $t$~channel. The top quark spin asymmetry, a related quantity to the top quark polarization, is extracted from the measured differential cross section. In a further step, the derivations of limits on anomalous couplings is demonstrated which characterize potential new physics models beyond the standard model in an effective approach.

In the second part, an early measurement of differential cross sections as a function of the top quark transverse momentum and rapidity is presented utilizing the first data recorded at a \acrlong{cm} energy of 13~\TeV with the \gls{cms} experiment in 2015. For this, the analysis strategy of the top quark polarization measurement has been significantly extended which resulted into various benefits. The measured cross sections are compared to the predictions of the single-top-quark spectra by various event generator programs. Lastly, prospects on extending this first measurement by including a significantly larger dataset recorded in 2016 are given.

The thesis is organized as follows. First a general introduction to the standard model is given in Ch.~\ref{ch:theory}. The phenomenology of top quark properties and its production mechanisms, relevant for this thesis, are introduced in Ch.~\ref{ch:top} with a particular emphasis on the electroweak coupling structure. The experimental setup consisting of the \gls{lhc} accelerator and the \gls{cms} experiment is outlined in Ch.~\ref{ch:exp} which is followed by a detailed description of the reconstruction of physics objects for analyses in Ch.~\ref{ch:exp}. In Ch.~\ref{ch:technique} the employed analysis techniques are elaborated which were used to perform the measurements. The differential cross section measurements at 8 and 13~\TeV are detailed in Ch.~\ref{ch:polarization} and Ch.~\ref{ch:diff13} respectively. Potential prospects on a future measurement have been studied in Ch.~\ref{ch:prospects}. The thesis is concluded in Ch.~\ref{ch:conclusion}.

In addition, an introduction to the emulation of track reconstruction in the fast simulation package of \gls{cms} can be found in the appendix (App.~\ref{ch:fsim}).

\section*{Unit convention}
\addcontentsline{toc}{section}{Unit convention}

The ``natural units'' of particle physics\footnote{Other fields of physics can have their own ``natural units'' which may not coincide with the definition used here.} are used throughout this thesis unless it is explicitly stated otherwise. These differ from the \glsmark{si} units and are derived by defining the natural constants as:

\begin{itemize}
\item speed of light: $\mathrm{c}\equiv 1$\,;
\item Planck constant: $\hbar\equiv 1$\,;
\item electric permittivity: $\epsilon_{0}\equiv 1$\,;
\item Boltzmann constant: $k_\mathrm{B}\equiv 1$\,.
\end{itemize}

This changes the units of the following quantities amongst others:

\begin{itemize}
\item length: $\big[\mathrm{m}\big]\rightarrow \big[\eV^{-1}\big]$\,;
\item time: $\big[\mathrm{s}\big]\rightarrow \big[\eV^{-1}\big]$\,;
\item mass: $\big[\mathrm{kg}\big]\rightarrow \big[\eV\big]$\,;
\item energy: $\big[\mathrm{J}\big]\rightarrow \big[\eV\big]$\,;
\item temperature: $\big[\mathrm{K}\big]\rightarrow \big[\eV\big]$\,.
\end{itemize}



