 \chapter{Conclusion}
\label{ch:conclusion}

First measurements of differential single-top-quark cross sections have been presented for which \acrlong{pp} collision data recorded with the \gls{cms} experiment at \acrlong{cm} energies of 8~and 13~TeV have been analyzed.

The production of polarized single top quarks in $t$~channel has been investigated through a measurement of the differential cross section as a function of the top quark polarization angle which is defined between the lepton from the top quark decay and the spectator quark recoiling against the W~boson in the top quark rest frame. The measurement is based on 8~\TeV collision data corresponding to 19.7~\invfb. Events with an isolated muon and two or three jets have been selected for the measurement while events containing single electrons have been studied as well as a cross check. The large contamination by events stemming from background processes mimicking the signature of single-top-quark events necessitated the usage of two \acrlongpl{bdt} for obtaining a signal-enriched phase space in data. From the resulting distribution the differential cross section is inferred at parton level. An asymmetry of $0.26\pm 0.11$ is extracted which is compatible within 2.0 standard deviation with the predicted asymmetry by the standard model of $0.44$. The measurement has been published in Ref.~\cite{Khachatryan:2015dzz}. In a further step, the derivation of limits on anomalous couplings have been illustrated using the measured asymmetry and related results.

An early measurement of differential single-top-quark cross sections at 13~TeV as a function of the top quark transverse momentum and rapidity has been performed as well. Events containing an isolated muon candidate and two or three jets have been selected from the first data at 13~TeV recorded in 2015 which corresponds to 2.3~\invfb. A novel fitting strategy has been developed which allows to extract the amount of signal events directly from single fits to the distribution of the transverse W~boson mass and a \gls{bdt} discriminant. The procedure does not require a signal-enriched region nor is a subtraction of background contributions from data distributions prior to unfolding necessary. The differential cross sections at parton level are inferred through unfolding directly for the estimated signal yields. The differential cross sections are compared to the predictions by various event generator programs. No significant deviation has been observed. The result has been published in Ref.~\cite{CMS-PAS-TOP-16-004}.

Lastly, various prospects on further measurements of differential single-top-quark cross sections at 13~TeV have been studied based on data corresponding to 36~\invfb recorded in 2016. In particular, the benefits for measuring differential cross sections at particle level have been investigated. A related study has been published in Ref.~\cite{particleStudies}.
