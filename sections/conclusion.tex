\chapter*{Conclusion}
\markboth{Conclusion}{Conclusion}
\addcontentsline{toc}{chapter}{Conclusion}

In this thesis, measurements of differential single-top-quark cross sections have been presented based on \acrlong{pp} collision data recorded with the \gls{cms} experiment at \acrlong{cm} energies of 8~and 13~TeV.

%In the \acrfull{sm} electroweak interactions between fermions and W~bosons are maximally parity violating. Only left-handed fermions or right-handed antifermions can participate in such interaction. In the \gls{sm} this feature is encoded as a V-A coupling structure.

In $t$-channel single-top-quark production the electroweak V-A coupling structure predicts the production of highly polarized top quarks. The degree of polarization is studied for the first time by measuring the cross section as a function of the polarization angle defined between the charged lepton from the top quark decay and a spectator quark which is produced in association with the single top quark. The measurement is based on 8~\TeV collision data corresponding to 19.7~\invfb. Events with an isolated muon and two or three jets have been selected for the measurement while events containing single electrons have been studied as a cross check. The large contamination by events stemming from background processes necessitated the usage of two \acrlongpl{bdt}. Their discriminants are used to estimate the amount of background events from data and to select data events in a signal-enriched phase space. The differential cross section is inferred at parton level though a regularized unfolding procedure from the distribution of the polarization angle in data after the remaining background contributions have been subtracted. The top quark spin asymmetry, a quantity related to the polarization, is extracted from the differential cross section through a linear fit. It is measured to be $0.26\pm 0.11$ which is compatible within 2.0 standard deviation with the expected \gls{sm} spin asymmetry of 0.44 at \gls{nlo}. The measurement has been published in Ref.~\cite{Khachatryan:2015dzz}. In a further step, the derivation of limits on anomalous couplings and the top quark polarization has been illustrated using the measured asymmetry and related results from the literature.

The first measurement of differential single-top-quark cross sections at 13~TeV as a function of the top quark transverse momentum and rapidity has been performed as well. Events containing an isolated muon candidate and two or three jets have been selected from the first data recorded in 2015 at the new \acrlong{cm} energy corresponding to 2.3~\invfb. A novel fitting strategy has been developed which allows to extract the amount of signal events as a function of the unfolding observables by performing multiple fits to the distributions of the transverse W~boson mass and a \gls{bdt} discriminant. The procedure does not require a signal-enriched region and neither is a subtraction of background contributions from the data distributions necessary before unfolding. The differential cross sections at parton level are instead inferred from the estimated signal yields through unfolding directly. The results are compared to the predictions by various event generator programs. No significant deviation has been observed. The measurement has been published in Ref.~\cite{CMS-PAS-TOP-16-004}.

Lastly, various improvements for further measurements of differential single-top-quark cross sections at 13~TeV have been studied using 36~\invfb of data recorded in 2016. In particular, the benefits for measuring differential cross sections at particle level have been investigated for which reduced migrations and a larger acceptance compared to the parton level are obtained which allows to achieve results with enhanced precision in the future. A related study has been published in Ref.~\cite{particleStudies}.
