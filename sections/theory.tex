\chapter{Theory and experimental status}

An introduction to the \newacronym{sm}{SM}{Standard Model of particle physics}\gls{sm}, its particle content, interactions, and properties is provided in the following. Special focus is attributed to the top quark which is the heaviest known fundamental particle to date. Here, latest experimental results are discussed and the theoretical foundation of the measurements within this thesis is laid out.


\section{General Quantum Field Theory of the Standard Model}

The \gls{sm} describes the interactions between fundamental particles. It is based on \newacronym{qft}{QFT}{Quantum Field Theory}\gls{qft} which allows to predict observables of particle interactions. Exemplary observables are production cross sections or lifetimes of unstable particles which can be calculated within the \gls{sm} \gls{qft} framework -- even fully automatized. The \gls{sm} validity is constantly challenged by measuring such observables using new experimental data. No significant deviations have been found so far that would hint towards physics \newacronym{bsm}{BSM}{Beyond the Standard Model}\gls{bsm}.


\subsection{Particle content}

Fundamental particles are objects for which experiments have revealed no internal structure so far. For example, the electron spatial radius has been limited to be smaller than $<10^{-18}~\mathrm{m}$~\cite{PhysRevLett.97.030801}. They are therefore considered as point-like. The particles can be grouped by their spin. Fermions with spin $\frac{1}{2}$ are listed in Tab.~\ref{tab:theory-particlecontent-fermions}. Bosons with spin $1$ are connected to fundamental interactions and act as their mediators as explained later~(Secs.~\ref{sec:theory-ewk} and~\ref{sec:theory-qcd}). They are listed in Tab.~\ref{tab:theory-particlecontent-bosons}

\mytable{\label{tab:theory-particlecontent-fermions}Particle content of the \gls{sm}.}{
\begin{tabular}{|c|c|c|c|c|}
\hline
\end{tabular}
}

\mytable{\label{tab:theory-particlecontent-bosons}Particle content of the \gls{sm}.}{
\begin{tabular}{|c|c|c|c|c|}
\hline
\end{tabular}
}

\subsection{General properties}
quantum fields, EW construction, gauge groups, symmetries, Noether currents, renormalization, running couplings
\subsection{Electroweak interactions}
\label{sec:theory-ewk}
chirality
\subsection{Higgs mechanism}
potential, CKM
\subsection{Strong interactions}
\label{sec:theory-qcd}
self-couplings, running coupling, nlo calculations
\subsection{Observables}
matrix elements, cross sections, decays, PDFs, angles (w polarizations)
\subsection{Open questions}
naturalness, gravity, gut, susy, dark matter

\section{The top quark}
\subsection{}
