%##############################################
\chapter{Theory and experimental status}
%##############################################

An introduction to the \gls{sm}, its particle content, interactions, and properties is provided in the following. Special focus is attributed to the top quark which is the heaviest known fundamental particle to date. Here, latest experimental results are discussed and the theoretical foundation of the measurements within this thesis is laid out.


%##############################################
\section{General Quantum Field Theory of the Standard Model}
%##############################################

The \gls{sm} describes the interactions between fundamental particles. It is based on \gls{qft} which allows to predict observables of particle interactions. Exemplary observables are production cross sections or lifetimes of unstable particles which can be calculated within its framework -- even fully automatized. The \gls{sm} validity is constantly challenged by comparing predictions to experimental data. No significant deviations have been found so far that would hint towards physics \gls{bsm}.


%##############################################
\subsection{Particle content}
%##############################################

Fundamental particles are objects for which experiments have revealed no internal structure. For example, only an upper limit on the spatial radius of an electron has been measured to be $<10^{-18}~\mathrm{m}$~\cite{PhysRevLett.97.030801}. Such particles are therefore considered as point-like. Fundamental particles can be grouped by their spin into fermions with spin~$\frac{1}{2}$ and bosons with integer spin. The fermions can be further divided into leptons and quarks where only the later can participate in strong interactions. Tables~\ref{tab:theory-leptons} and~\ref{tab:theory-quarks} list the leptons and quarks respectively. Each column is called a generation. It encapsulates an isospin pair whose components are therefore also referred to as up- or down-type respectively. It is unknown why there are exactly three lepton and three quark generations. Ordinary atoms consists of only particles from the first generator: electron, proton (uud) and neutron (udd).\todo{mention searches for 4th generation?}

\mytable{\label{tab:theory-leptons}Leptons of the \gls{sm}. Particle masses are taken from Ref.~\cite{Olive:2016xmw}. Uncertainties on the measured masses are omitted because the precision is beyond the sub permille level. For the neutrino masses only \glspl{cl} are given.}{
\begin{tabular}{r||c|c|c}
                        & 1. generation                 & 2. generation                 & 3. generation \\
    \hline
    \hline
    name                & electron ($\mathrm{e}^{-}$)   & muon ($\mu^{-}$)              & tau ($\tau^{-}$) \\
    \hline
    mass                & $511.0~\keV$                  & $105.66~\MeV$                 & $1.776~\GeV$ \\
    \hline
    electric charge     & $-1$                          & $-1$                          & $-1$ \\
    \hline
    isospin (\isoz)     & $\frac{1}{2}$                 & $\frac{1}{2}$                 & $\frac{1}{2}$ \\
    \hline
    \hline
    name                & electron neutrino             & muon neutrino                 & tau neutrino  \\
                        & ($\nu_\mathrm{e}$)            & ($\nu_{\mu}$)                 & ($\nu_{\tau}$) \\
    \hline
    mass                & $<225~\eV$                    & $<0.19~\MeV$                  & $<18.2~\MeV$ \\
                        & (95\% CL)                     & (90\% CL)                     & (95\%~CL)\\
    \hline
    electric charge     & 0                             & 0                             & 0 \\
    \hline
    isospin (\isoz)     & $-\frac{1}{2}$                & $-\frac{1}{2}$                & $-\frac{1}{2}$ \\
    \end{tabular}
}

For the neutrinos masses, only upper limits on their masses are given. Those are derived by combining measurements of beta decay kinematics with results from neutrino oscillation experiments. In the \gls{sm}, neutrinos are assumed to be massless. However, the observation\todo{need ref} of neutrino oscillations requires that at least two neutrinos have a non-zero mass.

\mytable{\label{tab:theory-quarks}Quarks of the \gls{sm}. For u,d,s,c,b quarks, \msbar masses are taken from Ref.~\cite{Olive:2016xmw}. The top quark pole mass is taken from Ref.~\cite{ATLAS:2014wva}.}{
\begin{tabular}{r||c|c|c}
                        & 1. generation                 & 2. generation                 & 3. generation \\
    \hline
    \hline
    name                & up (u)                        & charm (c)                     & top (t) \\
    \hline
    mass                & $2.2^{+0.6}_{-0.4}~\MeV$      & $1.27\pm0.03~\GeV$            & $173.34\pm0.71~\GeV$ \\
    \hline
    electric charge     & $\frac{2}{3}$                 & $\frac{2}{3}$                 & $\frac{2}{3}$ \\
    \hline
    isospin (\isoz)     & $\frac{1}{2}$                 & $\frac{1}{2}$                 & $\frac{1}{2}$ \\
    \hline
    \hline
    name                & down (d)                      & strange (s)                   & bottom (b)  \\
    \hline
    mass                & $4.7^{+0.5}_{-0.4}~\MeV$      & $96^{+8}_{-4}~\MeV$           & $4.18^{+0.04}_{-0.03}~\GeV$ \\
    \hline
    electric charge     & $-\frac{1}{3}$                & $-\frac{1}{3}$                & $-\frac{1}{3}$ \\
    \hline
    isospin (\isoz)     & $-\frac{1}{2}$                & $-\frac{1}{2}$                & $-\frac{1}{2}$ \\
    \end{tabular}
}


The bosons are connected to fundamental interactions as explained later in this chapter. They are listed in Tab.~\ref{tab:theory-bosons}. The Higgs boson is the only scalar particle~(spin~$0$) of the \gls{sm}. All other bosons carry a spin of~$1$.


\mytable{\label{tab:theory-bosons}Bosons of the \gls{sm}. Z and W boson masses are taken from Ref.~\cite{Olive:2016xmw}. The uncertainties on their masses are omitted because the precision is beyond the sub permille level. The Higgs mass is taken from Ref.~\cite{Aad:2015zhl}.}{
\begin{tabular}{r||c|c}
    name                & mass                 & associated interaction \\
    \hline
    \hline
    photon (\photon)    & $0$                  & electromagnetism \\
    \hline
    Z boson (\zboson)   & $91.19~\GeV$         & weak interaction \\
    \hline
    W boson (\wboson)   & $80.39~\GeV$         & weak interaction \\
    \hline
    Higgs boson (\higgs)& $125.09\pm0.24~\GeV$ & Yukawa interaction \\
    \hline  
    8 gluons (g)        & $0$                  & strong interaction \\                 
    \end{tabular}
}

Each fundamental particle has additionally a charge-conjugated partner called antiparticle. Other properties such as mass are identical. The photon, Z boson, and Higgs boson are their own antiparticle. It is still under study if the neutrino is its own antiparticle. Fermions with such a property are called Majorana particles~\cite{Majorana2006}. Experimentally, this can be probed in double $\beta$ decays where in the case of Majorana neutrinos the decay can occur without emitting two neutrinos. However, this scenario seems to be disfavored by recent results as reviewed in Ref.~\cite{Dell'Oro:2016dbc}.


%##############################################
\subsection{Quantum field theory}
%##############################################

In the framework of \gls{qft}, particles are described as excitation modes of quantized fields. This is also referred to as ``second quantization'' allowing to describe many-particle systems. Field operators can be decomposed as

\begin{align}
    \hat{\psi}(x)&=\sum_{i}^{\mathrm{N}}u_{i}(x)\hat{a}_{i} \\
    \hat{\psi}^{\dagger}(x)&=\sum_{i}^{\mathrm{N}}u^{\star}_{i}(x)\hat{a}^{\dagger}_{i},
\end{align}

where $u_{i}(x)$ denotes the ordinary wave function of a single particle and $\hat{a}^{\dagger}_{i}$ ($\hat{a}_{i}$) its creation (annihilation) operator, respectively.

As in classical mechanics, the action of a system can be expressed as

\begin{equation}
S=\int\mathrm{L}\,\mathrm{d}t=\iint\mathcal{L}\,\mathrm{d}^{3}\vec{x}\,\mathrm{dt}=\int\mathcal{L}\,\mathrm{d}^{4}x
\end{equation}

with the Lagrangian density $\mathcal{L}$. For example, a system of free fermions is described by the Dirac Lagrangian density,

\begin{equation}
\label{eq:theory-diracL}
\mathcal{L}_\mathrm{Dirac}=\bar{\psi}\big(i\gamma^\mu\partial_\mu-m\big)\psi
\end{equation}

using the definitions $\partial_\mu\equiv\partial/\partial x_\mu$ and $\bar{\psi}\equiv\psi^\dagger\gamma^{0}$ where $\gamma_\mu$ denote the Dirac matrices\footnote{Multiple representations are possible. The matrices need to satisfy a Clifford algebra with the anticommutation relation: $\big\{\gamma^\mu,\gamma^\nu\big\}=\gamma^\mu\gamma^\nu+\gamma^\nu\gamma^\mu=2g^{\mu\nu}$.}. The principle of least action, $\delta \mathrm{S}=0$, that is satisfied by the Euler-Lagrange equation yields the equation of motion as

\begin{equation}
\frac{\partial\mathcal{L}}{\partial\bar{\psi}}-\frac{\partial}{\partial_\mu}\Bigg(\frac{\partial\mathcal{L}}{\partial\big(\partial_\mu\bar{\psi}\big)}\Bigg)=\big(i\gamma^\mu\partial_\mu-m\big)\psi=0. \\
\end{equation}

Assuming $\psi\propto e^{-ip^{\mu}x_{\mu}}$ leads directly to the well-known energy-momentum relation

\begin{align}
0&=\big(i\gamma^\mu\partial_\mu-m\big)\cdot \big(-i\gamma^\nu\partial_\nu-m\big)\psi\\
 &=\big(\partial^{\mu}\partial_{\mu}+m^{2}\big)\psi \qquad \mathrm{(``Klein\mbox{-}Gordan''~equation)}  \\
 &\Rightarrow \big(-p^{\mu}p_{\mu}+m^{2}\big) = 0.
\end{align}


In the \gls{sm}, interactions between particles are introduced by requiring local invariance of the Lagrangian density for certain groups of gauge transformations. In the following, this is briefly demonstrated for the case of a $\mathrm{U(1)}$ transformation which leads to electromagnetic interactions.

The transformation of Eq.~\ref{eq:theory-diracL} using 

\begin{equation}
\psi(x)\mapsto\psi^{\prime}(x)=\psi\exp^{-iq\alpha(x)}
\end{equation}

where the phase $\alpha(x)$ depends on the local coordinates $x$ leads to

\begin{equation}
\mathcal{L}(\psi,\partial_\mu\psi)\mapsto\mathcal{L}(\psi^{\prime},\partial_\mu\psi^{\prime})=\bar{\psi}\big(i\gamma^\mu\partial_\mu+q\gamma^\mu\partial_\mu\alpha(x)-m\big)\psi
\end{equation}

The invariance $\mathcal{L}(\psi,\partial_\mu\psi)=\mathcal{L}(\psi^{\prime},\partial_\mu\psi^{\prime})$ is restored by adding a bosonic spin-1 field $A_{\mu}(x)$ which interacts with $\psi$ while transforming as

\begin{equation}
A_{\mu}(x)\mapsto A^{\prime}_{\mu}(x)=A_\mu(x)-\partial_\mu\alpha(x).
\end{equation}

This procedure yields a Lagrangian density containing the following terms:

\begin{align}
\mathcal{L}=~~&\bar{\psi}\big(i\gamma^\mu\partial_\mu-m\big)\psi &(\mathrm{fermion~propagator}) \\
            +&q\bar{\psi}\gamma_{\mu}\psi A_{\mu} &(\mathrm{interaction}) \label{eq:theory-EM-int}\\
            -&\tfrac{1}{4}\big(\partial_\mu A_\nu-\partial_\nu A_\mu\big)^{2} &(\mathrm{boson~propagator})
\end{align}

The boson propagator describing the dynamics of a free $A_\mu$ field has been additionally added to the Lagrangian density. It is already invariant under the local gauge transformation. The introduced interaction~(Eq.~\ref{eq:theory-EM-int}) which is required to ensure the invariance under $\mathrm{U(1)}$ transformation can be identified as electromagnetic interaction between a fermion described by $\psi$ with electric charge $q$ and a photon described by $A_\mu$. Furthermore, the photon is predicted to be massless since adding a term of the form $m^{2}_{A}A^\mu A_\mu$ would violate the invariance.

Other interactions of the \gls{sm} are also connected to local gauge transformations which can be introduced through similar procedures. A common property follows from the Noether theorem which states that for each continuous transformation a conserved current exists. Hence the charge associated to each gauge group is conserved. This would however be already the case for a global transformation. The requirement of invariance under local gauge transformations is a puzzling feature of the theory. \todo{ref to some philosophical arguments? is there an implication from renormalization?}

%##############################################
\subsection{Electroweak interactions and Higgs mechanism}
%##############################################
\label{sec:theory-ewk}

Electromagnetism and weak interactions can be unified using a $\mathrm{U(1)}\otimes \mathrm{SU(2)}$ gauge group. Furthermore, the theory needs to account for the following features found experimentally.

\begin{itemize}

\item Parity is not conserved in weak interactions. Such interactions depend on the spin being aligned towards or against the momentum of a particle. Experimentally, the Wu experiment~\cite{PhysRev.105.1413} discovered this feature in ${}_{27}^{60}\mathrm{Co}\to{}_{28}^{60}\mathrm{Ni}+e^{-}\bar{\nu}_{e}\gamma\gamma$ decays by analyzing the direction of the electron with respect to the polarization of the cobalt probe through an external magnetic field. 

\item The UA1 and UA2 experiments at the CERN SPS Proton-Antiproton Collider discovered that \wboson bosons~\cite{Arnison:1983rp,Banner:1983jy} and \zboson bosons~\cite{Arnison:1983mk,Bagnaia:1983zx} -- mediators of weak interactions -- are massive. The masses of these particles have to be introduced in a different way if the concept of local gauge invariance should continue to hold.

\end{itemize}

To account for the parity violation, fermion fields have to be first decomposed into chiral eigenstates using the projections

\begin{align}
\psi_\mathrm{L}&\equiv\mathrm{P}_\mathrm{L}\psi=\tfrac{1}{2}(1+\gamma_{5})\psi \\
\psi_\mathrm{R}&\equiv\mathrm{P}_\mathrm{R}\psi=\tfrac{1}{2}(1-\gamma_{5})\psi
\end{align}

with $\gamma_{5}=i\gamma_{0}\gamma_{1}\gamma_{2}\gamma_{3}$\footnote{Properties: $(\gamma_{5})^{\dagger}=\gamma_{5}$; ~~$(\gamma_{5})^2=\mathrm{I}_\mathrm{4x4}$; ~~ $\{\gamma_{5},\gamma_{\mu}\}=\gamma_{5}\gamma_{0}+\gamma_{0}\gamma_{5}=0$.} where $\psi_\mathrm{L}$ ($\psi_\mathrm{R}$) is called ``left-handed'' (``right-handed'') respectively. In the case of massless particles, Eq.~\ref{eq:theory-diracL} decouples into two separate equations for $\psi=\psi_\mathrm{L}+\psi_\mathrm{R}$. In this case the chirality is equal to the helicity

\begin{equation}
\mathrm{H}\equiv\frac{\vec{p}\cdot\vec{s}}{|\vec{p}|}
\end{equation}

which denotes the aligned towards~($H=1$) or against~($H=-1$) of the spin $\vec{s}$ projected onto the momentum axis. For massive particles, Eq.~\ref{eq:theory-diracL} does not decouple. \todo{later this is not invariant under SU2 transformation}

The Glawhow-Weinberg-Salam model~\cite{Salam:1964ry,Weinberg:1967tq,Glashow:1961tr} splits the fermion fields into ``left-handed'' doublets 

\begin{align}
\vec{E}_\mathrm{L}&=\Bigg\{\colvec{2}{e^{-}_\mathrm{L}}{\nu_\mathrm{e,L}},\colvec{2}{\mu^{-}_\mathrm{L}}{\nu_{\mu,\mathrm{L}}},\colvec{2}{\tau^{-}_\mathrm{L}}{\nu_{\tau,\mathrm{L}}}\Bigg\} \\
\vec{Q}_\mathrm{L}&=\Bigg\{\colvec{2}{\mathrm{u}_\mathrm{L}}{\mathrm{d}_\mathrm{L}},\colvec{2}{\mathrm{c}_\mathrm{L}}{\mathrm{s}_\mathrm{L}},\colvec{2}{\mathrm{t}_\mathrm{L}}{\mathrm{b}_\mathrm{L}}\Bigg\}
\end{align}

and ``right-handed'' singlets 

\begin{align}
\vec{e}_\mathrm{R}&=\big\{\mathrm{e}^{-}_\mathrm{R},\mu^{-}_\mathrm{R},\tau^{-}_\mathrm{R}\big\} \\
\vec{u}_\mathrm{R}&=\big\{\mathrm{u}_\mathrm{R},\mathrm{c}_\mathrm{R},\mathrm{t}_\mathrm{R}\big\} \\
\vec{d}_\mathrm{R}&=\big\{\mathrm{d}_\mathrm{R},\mathrm{s}_\mathrm{R},\mathrm{b}_\mathrm{R}\big\}
\end{align}

of the $\mathrm{SU(2)}$ group. Right-handed neutrinos do not participate in any interaction within the \gls{sm}. 

Higgs potential, yukawa interaction, mass term CKM

%##############################################
\subsection{Strong interactions}
%##############################################
\label{sec:theory-qcd}
self-couplings, running coupling, nlo calculations

%##############################################
\subsection{Observables}
%##############################################
matrix elements, cross sections, decays, PDFs, angles (w polarizations), Noether currents, renormalization

%##############################################
\subsection{Open questions}
%##############################################
naturalness, gravity, gut, susy, dark matter

%##############################################
\section{The top quark}
%##############################################
\subsection{}
