%##############################################
\chapter{Theory and experimental status}
%##############################################

An introduction to the \gls{sm}, its particle content, interactions, and properties is provided in the following. Special focus is attributed to the top quark which is the heaviest known fundamental particle to date. Here, latest experimental results are discussed and the theoretical foundation of the measurements within this thesis is laid out.


%##############################################
\section{General Quantum Field Theory of the Standard Model}
%##############################################

The \gls{sm} describes the interactions between fundamental particles. It is based on \gls{qft} which allows to predict observables of particle interactions. Exemplary observables are production cross sections or lifetimes of unstable particles which can be calculated within its framework -- even fully automatized. The \gls{sm} validity is constantly challenged by comparing predictions to experimental data. No significant deviations have been found so far that would hint towards physics \gls{bsm}.


%##############################################
\subsection{Particle content}
%##############################################

Fundamental particles are objects for which experiments have revealed no internal structure. For example, only an upper limit on the spatial radius of an electron has been measured to be $<10^{-18}~\mathrm{m}$~\cite{PhysRevLett.97.030801}. Such particles are therefore considered as point-like. Fundamental particles can be grouped by their spin into fermions with spin~$\frac{1}{2}$ and bosons with integer spin. The fermions can be further divided into leptons and quarks where only the later can participate in strong interactions. Tables~\ref{tab:theory-leptons} and~\ref{tab:theory-quarks} list the leptons and quarks respectively. Each column is called a generation. It encapsulates an isospin pair whose components are therefore also referred to as up- or down-type respectively. It is unknown why there are exactly three lepton and three quark generations.\todo{mention searches for 4th generation?}

\mytable{\label{tab:theory-leptons}Leptons of the \gls{sm}. Particle masses are taken from Ref.~\cite{Olive:2016xmw}. Uncertainties on the measured masses are omitted because the precision is beyond the sub permille level. For the neutrino masses only \glspl{cl} are given.}{
\begin{tabular}{r||c|c|c}
                        & 1. generation                 & 2. generation                 & 3. generation \\
    \hline
    \hline
    name                & electron (e)                  & muon ($\mu$)                  & tau ($\tau$) \\
    \hline
    mass                & $511.0~\keV$                  & $105.66~\MeV$                 & $1.776~\GeV$ \\
    \hline
    electric charge     & $-1$                          & $-1$                          & $-1$ \\
    \hline
    isospin (\isoz)     & $\frac{1}{2}$                 & $\frac{1}{2}$                 & $\frac{1}{2}$ \\
    \hline
    \hline
    name                & electron neutrino             & muon neutrino                 & tau neutrino  \\
                        & ($\nu_{e}$)                   & ($\nu_{\mu}$)                 & ($\nu_{\tau}$) \\
    \hline
    mass                & $<225~\eV$                    & $<0.19~\MeV$                  & $<18.2~\MeV$ \\
                        & (95\% CL)                     & (90\% CL)                     & (95\%~CL)\\
    \hline
    electric charge     & 0                             & 0                             & 0 \\
    \hline
    isospin (\isoz)     & $-\frac{1}{2}$                & $-\frac{1}{2}$                & $-\frac{1}{2}$ \\
    \end{tabular}
}

For the neutrinos masses, only upper limits on their masses are given. Those are derived by combining measurements of beta decay kinematics with results from neutrino oscillation experiments. In the \gls{sm}, neutrinos are assumed to be massless.

\mytable{\label{tab:theory-quarks}Quarks of the \gls{sm}. For u,d,s,c,b quarks, \msbar masses are taken from Ref.~\cite{Olive:2016xmw}. The top quark pole mass is taken from Ref.~\cite{ATLAS:2014wva}.}{
\begin{tabular}{r||c|c|c}
                        & 1. generation                 & 2. generation                 & 3. generation \\
    \hline
    \hline
    name                & up (u)                        & charm (c)                     & top (t) \\
    \hline
    mass                & $2.2^{+0.6}_{-0.4}~\MeV$      & $1.27\pm0.03~\GeV$            & $173.34\pm0.71~\GeV$ \\
    \hline
    electric charge     & $\frac{2}{3}$                 & $\frac{2}{3}$                 & $\frac{2}{3}$ \\
    \hline
    isospin (\isoz)     & $\frac{1}{2}$                 & $\frac{1}{2}$                 & $\frac{1}{2}$ \\
    \hline
    \hline
    name                & down (d)                      & strange (s)                   & bottom (b)  \\
    \hline
    mass                & $4.7^{+0.5}_{-0.4}~\MeV$      & $96^{+8}_{-4}~\MeV$           & $4.18^{+0.04}_{-0.03}~\GeV$ \\
    \hline
    electric charge     & $-\frac{1}{3}$                & $-\frac{1}{3}$                & $-\frac{1}{3}$ \\
    \hline
    isospin (\isoz)     & $-\frac{1}{2}$                & $-\frac{1}{2}$                & $-\frac{1}{2}$ \\
    \end{tabular}
}


The bosons are connected to fundamental interactions as explained later in this chapter. They are listed in Tab.~\ref{tab:theory-bosons}. The Higgs boson is the only scalar particle~(spin~$0$) of the \gls{sm}. All other bosons carry a spin of~$1$.


\mytable{\label{tab:theory-bosons}Bosons of the \gls{sm}. Z and W boson masses are taken from Ref.~\cite{Olive:2016xmw}. The uncertainties on their masses are omitted because the precision is beyond the sub permille level. The Higgs mass is taken from Ref.~\cite{Aad:2015zhl}.}{
\begin{tabular}{r|c|c}
    name                & mass                 & associated interactions \\
    \hline
    \hline
    photon ($\gamma$)   & $0$                  & electromagnetism \\
    \hline
    Z boson (Z)         & $91.19~\GeV$         & weak interaction \\
    \hline
    W boson ($W^{\pm}$) & $80.39~\GeV$         & weak interaction \\
    \hline
    Higgs boson (H)     & $125.09\pm0.24~\GeV$ & Yukawa interaction \\
    \hline  
    8 gluons (g)        & $0$                  & strong interaction \\                 
    \end{tabular}
}

Each fundamental particle has additionally a charge-conjugated partner called antiparticle. Other properties such as mass are identical. The photon, Z boson, and Higgs boson are their own antiparticle. It is still under study if the neutrino is its own antiparticle. Fermions with such a property are called Majorana particles~\cite{Majorana2006}. Experimentally, this can be probed in double $\beta$ decays where in the case of Majorana neutrinos the decay can occur without emitting two neutrinos. However, this scenario seems to be disfavored by recent results as reviewed in Ref.~\cite{Dell'Oro:2016dbc}.


%##############################################
\subsection{General properties}
%##############################################

In the framework of \gls{qft}, particles are viewed as excitation modes of quantized fields. This is also referred to as ``second quantization'' which allows to describe many-particle systems. Field operators can be decomposed as

\begin{align}
    \hat{\Psi}(x)&=\sum_{i}^{\mathrm{N}}u_{i}(x)\hat{a}_{i} \\
    \hat{\Psi}^{\dagger}(x)&=\sum_{i}^{\mathrm{N}}u^{\star}_{i}(x)\hat{a}^{\dagger}_{i},
\end{align}

where $u_{i}(x)$ denotes the ordinary wave function of a single particle and $\hat{a}^{\dagger}_{i}$ ($\hat{a}_{i}$) its creation (annihilation) operator, respectively.


Lagrangian, EW construction, gauge groups, symmetries, Noether currents, renormalization, running couplings
\subsection{Electroweak interactions}
\label{sec:theory-ewk}
chirality
\subsection{Higgs mechanism}
potential, yukawa interaction, mass term CKM
\subsection{Strong interactions}
\label{sec:theory-qcd}
self-couplings, running coupling, nlo calculations
\subsection{Observables}
matrix elements, cross sections, decays, PDFs, angles (w polarizations)
\subsection{Open questions}
naturalness, gravity, gut, susy, dark matter

\section{The top quark}
\subsection{}
