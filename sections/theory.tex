\chapter{Theory and experimental status}

An introduction to the \gls{sm}, its particle content, interactions, and properties is provided in the following. Special focus is attributed to the top quark which is the heaviest known fundamental particle to date. Here, latest experimental results are discussed and the theoretical foundation of the measurements within this thesis is laid out.


\section{General Quantum Field Theory of the Standard Model}

The \gls{sm} describes the interactions between fundamental particles. It is based on \gls{qft} which allows to predict observables of particle interactions. Exemplary observables are production cross sections or lifetimes of unstable particles which can be calculated within its framework -- even fully automatized. The \gls{sm} validity is constantly challenged by comparing predictions to experimental data. No significant deviations have been found so far that would hint towards physics \gls{bsm}.


\subsection{Particle content}

Fundamental particles are objects for which experiments have revealed no internal structure. For example, the spatial radius of an electron has been limited to be smaller than $<10^{-18}~\mathrm{m}$~\cite{PhysRevLett.97.030801}. Such particles are therefore considered as point-like. Fundamental particles can be grouped by their spin into fermions with spin~$\frac{1}{2}$ and bosons with integer spin. The fermions can be further divided into leptons and quarks where only the later can participate in strong interactions. Tables~\ref{tab:theory-leptons} and~\ref{tab:theory-quarks} list the leptons and quarks respectively. Each column encapsulating an isospin pair~(up/down) of fermions is called a generation. It is unknown why there are exactly three lepton and three quark generations.

\mytable{\label{tab:theory-leptons}Leptons of the \gls{sm}. Particle masses are taken from Ref.~\cite{Olive:2016xmw}. Only \glspl{cl} are given for the neutrino masses.}{
\begin{tabular}{r||c|c|c}
                    & 1. generation                  & 2. generation                  & 3. generation \\
\hline
\hline
name                & electron (e)                  & muon ($\mu$)                  & tau ($\tau$) \\
\hline
mass                & $511.0~\keV$                  & $105.66~\MeV$                 & $1.776~\GeV$ \\
\hline
electric charge     & $-1$                          & $-1$                          & $-1$ \\
\hline
isospin (\isoz)     & $\frac{1}{2}$                 & $\frac{1}{2}$                 & $\frac{1}{2}$ \\
\hline
\hline
name                & electron neutrino             & muon neutrino                 & tau neutrino  \\
                    & ($\nu_{e}$)                   & ($\nu_{\mu}$)                 & ($\nu_{\tau}$) \\
\hline
mass                & $<225~\eV$                    & $<0.19~\MeV$                  & $<18.2~\MeV$ \\
                    & (95\% CL)                     & (90\% CL)                     & (95\%~CL)\\
\hline
electric charge     & 0                             & 0                             & 0 \\
\hline
isospin (\isoz)     & $-\frac{1}{2}$                & $-\frac{1}{2}$                & $-\frac{1}{2}$ \\
\end{tabular}
}

\mytable{\label{tab:theory-quarks}Quarks of the \gls{sm}.}{
\begin{tabular}{|c|c|c|c|c|}
\hline
\end{tabular}
}

For the neutrinos, only limits on


The bosons are connected to fundamental interactions and act as their mediators as explained later~(Secs.~\ref{sec:theory-ewk} and~\ref{sec:theory-qcd}). They are listed in Tab.~\ref{tab:theory-bosons}. The Higgs boson is the only scalar particle~(spin~$0$) of the \gls{sm}. All other bosons carry a spin of~$1$.


\mytable{\label{tab:theory-bosons}Bosons of the \gls{sm}.}{
\begin{tabular}{|c|c|c|c|c|}
\hline
\end{tabular}
}


Antiparticles


\subsection{General properties}
quantum fields, EW construction, gauge groups, symmetries, Noether currents, renormalization, running couplings
\subsection{Electroweak interactions}
\label{sec:theory-ewk}
chirality
\subsection{Higgs mechanism}
potential, CKM
\subsection{Strong interactions}
\label{sec:theory-qcd}
self-couplings, running coupling, nlo calculations
\subsection{Observables}
matrix elements, cross sections, decays, PDFs, angles (w polarizations)
\subsection{Open questions}
naturalness, gravity, gut, susy, dark matter

\section{The top quark}
\subsection{}
