\chapter{Theory and experimental status}

The \newacronym{sm}{SM}{Standard Model of particle physics}\gls{sm} describes the interactions between fundamental particles. It is based on \newacronym{qft}{QFT}{Quantum Field Theory}\gls{qft} which allows to predict observables of particle interactions. Exemplary, cross sections or the lifetime and decay channels of unstable fundamental particles can be calculated within the SM framework even fully automatized. By measuring a plethora of observables, the SM validity is constantly challenged with new experimental data. No significant deviations have been found so far that would hint towards physics \newacronym{bsm}{BSM}{Beyond the Standard Model}\gls{bsm}.

In the following, an introduction to the SM, its particle content, interactions, and properties is provided. Special focus is attributed to the top quark which is the heaviest known fundamental particle to date. Here, latest experimental results are discussed and the theoretical foundation of the measurements within this thesis is laid out.

\section{General Quantum Field Theory of the Standard Model}
\subsection{Particle content}
\subsection{General properties}
quantum fields, EW construction, gauge groups, symmetries, Noether currents, renormalization, running couplings
\subsection{Electroweak interactions}
chirality
\subsection{Higgs mechanism}
potential, CKM
\subsection{Strong interactions}
self-couplings, running coupling, nlo calculations
\subsection{Observables}
matrix elements, cross sections, decays, PDFs
\subsection{Open questions}
naturalness, gravity, gut, susy, dark matter

\section{The top quark}
\subsection{}
